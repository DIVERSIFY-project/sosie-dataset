Listing \ref{lst:nextLong} displays an excerpt of the \texttt{nextLong()} method in GSon. 
Under certain conditions, the method returns a value, skipping all the computation performed after the \texttt{try} block. 
Yet, this is a shortcut in the computation, i.e., if we remove the \texttt{return} statement, the subsequent code will return exactly the same value, with some additional computation. 

%com.google.json.stream.JsonReader:957
%source:"Sosies-GSON-21-08-2015.txt" in dropbox
\begin{minipage}{\columnwidth}
\begin{lstlisting}[caption={\texttt{nextLong} in GSon and a sosie},label={lst:nextLong},language=java,numbers=left]
//original
public long nextLong() throws IOException {
  *{\color{grey}...}*)
  if (p == PEEKED_NUMBER) {
    peekedString = new String(buffer, pos, peekedNumberLength);
    pos += peekedNumberLength;
  } else if (p == PEEKED_SINGLE_QUOTED || p == PEEKED_DOUBLE_QUOTED) {
    peekedString = nextQuotedValue(p == PEEKED_SINGLE_QUOTED ? '\'' : '"');
    try {
      long result = Long.parseLong(peekedString);
      peeked = PEEKED_NONE;
      pathIndices[stackSize - 1]++;
      return result; 
      } 
  *{\color{grey}...}*)
  return result;}
//sosie
public long nextLong() throws IOException {
  *{\color{grey}...}*)
  if (p == PEEKED_NUMBER) {
    peekedString = new String(buffer, pos, peekedNumberLength);
    pos += peekedNumberLength;
  } else if (p == PEEKED_SINGLE_QUOTED || p == PEEKED_DOUBLE_QUOTED) {
    peekedString = nextQuotedValue(p == PEEKED_SINGLE_QUOTED ? '\'' : '"');
    try {
      long result = Long.parseLong(peekedString);
      peeked = PEEKED_NONE;
      pathIndices[stackSize - 1]++;
      *{\textbf{\color{orange}assignment deleted}}*)
      } 
  *{\color{grey}...}*)
  return result;}
\end{lstlisting}
\tabcolsep=0.11cm
\begin{tabular}{>{\small}c>{\small}c>{\small}c>{\small}c>{\small}c>{\small}c>{\small}c>{\small}c}
\hline
\rowcolor{lightgray} \#tc & \#assert & transfo & node & min & max & median & mean   \\
\rowcolor{lightgray}  & & type & type & depth  & depth & depth & depth  \\ 
\hline
&  & del &  &  &  &  & \\
\hline
\end{tabular}
\end{minipage}
%*{\textbf{\color{orange}assignment deleted}}*)
%*{\color{grey}...}*)