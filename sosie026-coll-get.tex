Listing \ref{lst:utils-get} illustrates two kinds of plasticity exploited by the sosie. 
First, it exploits the implementation redundancy inside the program, i.e., \texttt{((Object[]) object)[i]} and \texttt{Array.get(object, i)} have the same behavior, yet are based on completely different implementations. 
Second, it also introduced extra checking code, as well as failure diversity.

%org.apache.commons.collections4.CollectionUtils:1260
\begin{figure}[ht]
\begin{lstlisting}[caption={\texttt{get} in commons.collection},label={lst:utils-get},language=java,numbers=left]
//original program
public static Object get(final Object object, final int index) {
  int i = index;
  if (i < 0) {
    throw new IndexOutOfBoundsException("Index cannot be negative: " + i);
  }
  if (object instanceof Map<?,?>) {
    final Map<?, ?> map = (Map<?, ?>) object;
    final Iterator<?> iterator = map.entrySet().iterator();
    return get(iterator, i);
  } else if (object instanceof Object[]) {
    return ((Object[]) object)[i];
    } else if (object instanceof Iterator<?>) {
      final Iterator<?> it = (Iterator<?>) object;
      while (it.hasNext()) {
        i--;
        if (i == -1) {return it.next();}
        it.next();
      }
      throw new IndexOutOfBoundsException("Entry does not exist: " + i);
    } else if (object instanceof Collection<?>) {
      final Iterator<?> iterator = ((Collection<?>) object).iterator();
      return get(iterator, i);
    } else if (object instanceof Enumeration<?>) {
        final Enumeration<?> it = (Enumeration<?>) object;
        while (it.hasMoreElements()) {
          i--;
          if (i == -1) {return it.nextElement();} 
          else {it.nextElement();}
        }
        throw new IndexOutOfBoundsException("Entry does not exist: " + i);
    } else if (object == null) {
      throw new IllegalArgumentException("Unsupported object type: null");
    } else {
      try {return Array.get(object, i);
      } catch (final IllegalArgumentException ex) {
        throw new IllegalArgumentException("Unsupported object type: " + object.getClass().getName());
      }
    }
}
// sosie  
public static Object get(final Object object, final int index) {
  int i = index;
  if (i < 0) {
    throw new IndexOutOfBoundsException("Index cannot be negative: " + i);
  }
  if (object instanceof Map<?,?>) {
    final Map<?, ?> map = (Map<?, ?>) object;
    final Iterator<?> iterator = map.entrySet().iterator();
    return get(iterator, i);
  } else if (object instanceof Object[]) {
    %*{\textbf{\color{orange}the return statement is replaced by the following}}*)
    try {
        return Array.get(object, i);
    } catch (final IllegalArgumentException ex) {
        throw new IllegalArgumentException("Unsupported object type: " + object.getClass().getName());
    }
    } else if (object instanceof Iterator<?>) {
      final Iterator<?> it = (Iterator<?>) object;
      while (it.hasNext()) {
        i--;
        if (i == -1) {return it.next();}
        it.next();
      }
      throw new IndexOutOfBoundsException("Entry does not exist: " + i);
    } else if (object instanceof Collection<?>) {
      final Iterator<?> iterator = ((Collection<?>) object).iterator();
      return get(iterator, i);
    } else if (object instanceof Enumeration<?>) {
        final Enumeration<?> it = (Enumeration<?>) object;
        while (it.hasMoreElements()) {
          i--;
          if (i == -1) {return it.nextElement();} 
          else {it.nextElement();}
        }
        throw new IndexOutOfBoundsException("Entry does not exist: " + i);
    } else if (object == null) {
      throw new IllegalArgumentException("Unsupported object type: null");
    } else {
      try {return Array.get(object, i);
      } catch (final IllegalArgumentException ex) {
        throw new IllegalArgumentException("Unsupported object type: " + object.getClass().getName());
      }
    }
}
\end{lstlisting}
\tabcolsep=0.11cm
\begin{tabular}{>{\small}c>{\small}c>{\small}c>{\small}c>{\small}c>{\small}c>{\small}c>{\small}c}
\hline
\rowcolor{lightgray} \#tc & \#assert & transfo & node & min & max & median & mean   \\
\rowcolor{lightgray}  & & type & type & depth  & depth & depth & depth  \\ 
\hline
 &  & rep &  &  &  &  & \\
\hline
\end{tabular}
\end{figure}