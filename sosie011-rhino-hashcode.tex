Listing \ref{lst:hashcode} is a good example of a sosie in a function which specification is flexible, i.e., it exhibits specification diversity. 
\todo{elaborate on the expected behavior of hashcode}

% position: org.mozilla.classfile.FieldOrMethodRef:4776
\begin{minipage}{\columnwidth}
\begin{lstlisting}[caption={\texttt{hashCode} in Rhino and a sosie},label={lst:hashcode},language=java,numbers=left]
//original
public int hashCode(){}
  if (hashCode == -1) {
  int h1 = className.hashCode();
  int h2 = name.hashCode();
  int h3 = type.hashCode();
  hashCode = h1 ^ h2 ^ h3;
  }
  return hashCode;
}
//sosie
public int hashCode(){}
  if (hashCode == -1) {
  int h1 = className.length();
  int h2 = name.hashCode();
  int h3 = type.hashCode();
  hashCode = h1 ^ h2 ^ h3;
  }
  return hashCode;
}
\end{lstlisting}
\tabcolsep=0.11cm
\begin{tabular}{>{\small}c>{\small}c>{\small}c>{\small}c>{\small}c>{\small}c>{\small}c>{\small}c}
\hline
\rowcolor{lightgray} \#tc & \#assert & transfo & node & min & max & median & mean   \\
\rowcolor{lightgray}  & & type & type & depth  & depth & depth & depth  \\ 
\hline
&  & rep &  &  &  &  & \\
\hline
\end{tabular}
\end{minipage}