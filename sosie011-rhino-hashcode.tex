Some program regions implement behavior which specfication is intrinsically plastic, i.e., behaviors for which the correctness of the output is not binary (one single possible correct value, all others are wrong) but has to be in a range of values. 
These regions are extremely forgiving and provide great opportunities for sosiefication since it can transform the programs in many ways and still produce valuable functionality, as long as the outcome remains in a range. 

One situation that we have encountered many times relates to the production of hash keys.
Methods that produce these keys have a very plastic specification: they must return an integer value that can be used to identify an element. There is no constraint on the value of the key, but there is one constraint on the behavior of the hash function: it must be deterministic.
Listing \ref{lst:hashcode} is an example of such a sosie.

% position: org.mozilla.classfile.FieldOrMethodRef:4776
\begin{minipage}{\columnwidth}
\begin{lstlisting}[caption={\texttt{hashCode} in Rhino and a sosie},label={lst:hashcode},language=java,numbers=left]
//original
public int hashCode(){}
  if (hashCode == -1) {
  int h1 = className.hashCode();
  int h2 = name.hashCode();
  int h3 = type.hashCode();
  hashCode = h1 ^ h2 ^ h3;
  }
  return hashCode;
}
//sosie
public int hashCode(){}
  if (hashCode == -1) {
  int h1 = className.length();
  int h2 = name.hashCode();
  int h3 = type.hashCode();
  hashCode = h1 ^ h2 ^ h3;
  }
  return hashCode;
}
\end{lstlisting}
\tabcolsep=0.11cm
\begin{tabular}{>{\small}c>{\small}c>{\small}c>{\small}c>{\small}c>{\small}c>{\small}c>{\small}c}
\hline
\rowcolor{lightgray} \#tc & \#assert & transfo & node & min & max & median & mean   \\
\rowcolor{lightgray}  & & type & type & depth  & depth & depth & depth  \\ 
\hline
&  & rep &  &  &  &  & \\
\hline
\end{tabular}
\end{minipage}