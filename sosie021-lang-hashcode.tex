Listing \ref{lst:range-hashCode} is an example where sosiefication exploits plasticity in the computation, which can be found in many programs. 
The \texttt{hashCode()} method must return an integer value that can be used to quickly retrieve a value in a collection. Yet, the exact value of this integer is not part of the specification, i.e., there are many ways to compute this value. 
Thus, removing a statement in this method does not change the validity of the service provided by the function. 

%org.apache.commons.lang.Range:433
%source:"plastic numeric function.txt" in dropbox
\begin{minipage}{\columnwidth}
\begin{lstlisting}[caption={\texttt{hashCode} in commons.lang and a sosie},label={lst:range-hashCode},language=java,numbers=left]
//original
public int hashCode() {
  int result = hashCode;
  if (hashCode == 0) {
    result = 17;
    result = 37 * result + getClass().hashCode();
    result = 37 * result + minimum.hashCode();
    result = 37 * result + maximum.hashCode();
    hashCode = result;
  }
  return result;}
//sosie
public int hashCode() {
  int result = hashCode;
  if (hashCode == 0) {
    result = 17;
    result = 37 * result + getClass().hashCode();
    %*{\textbf{\color{orange}\sout{assignment deleted}}*)
    result = 37 * result + maximum.hashCode();
    hashCode = result;
  }
  return result;}
\end{lstlisting}
\tabcolsep=0.11cm
\begin{tabular}{>{\small}c>{\small}c>{\small}c>{\small}c>{\small}c>{\small}c>{\small}c>{\small}c}
\hline
\rowcolor{lightgray} \#tc & \#assert & transfo & node & min & max & median & mean   \\
\rowcolor{lightgray}  & & type & type & depth  & depth & depth & depth  \\ 
\hline
1&  & del & stmt list &1  &1  &1  &1 \\
\hline
\end{tabular}
\end{minipage}