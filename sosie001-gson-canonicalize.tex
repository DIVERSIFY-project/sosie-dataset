Listing \ref{lst:canonicalize} is one extreme case found in Google's GSon library (v. 2.3). 
The sosie completely removes the body of the method, which is supposed to transform the type passed as parameter into an equivalent version that is serializable, and instead it returns the parameter. 
The sosie is covered by 624 different test cases, it is executed 6000 times and all executions complete succesfully and all assertions in the test cases are satisfied. 
This is an example of an advanced feature implemented in the core part of GSon that is not necessary to make the library run correctly.

%https://github.com/google/gson/blob/gson-2.3/src/main/java/com/google/gson/internal/$Gson$Types.java:94
\begin{minipage}{\columnwidth}
\begin{lstlisting}[caption={\texttt{canonicalize} in EasyMock and a sosie},label={lst:canonicalize},language=java,numbers=left]
//original
public static Type canonicalize(Type type) {
  if (type instanceof Class) {
    Class<?> c = (Class<?>) type;
    return c.isArray() ? new GenericArrayTypeImpl(canonicalize(c.getComponentType())) : c;
    } 
  else 
    if (type instanceof ParameterizedType) {
      ParameterizedType p = (ParameterizedType) type;
      return new ParameterizedTypeImpl(p.getOwnerType(),p.getRawType(), p.getActualTypeArguments());
    } 
    else 
      if (type instanceof GenericArrayType) {
        GenericArrayType g = (GenericArrayType) type;
        return new GenericArrayTypeImpl(g.getGenericComponentType());
      } 
      else 
      if (type instanceof WildcardType) {
        WildcardType w = (WildcardType) type;
        return new WildcardTypeImpl(w.getUpperBounds(), w.getLowerBounds());
      } 
      else {
        return type;
      }
}
//sosie
public static Type canonicalize(Type type) {
  return type;
}
\end{lstlisting}
\tabcolsep=0.11cm
\begin{tabular}{>{\small}c>{\small}c>{\small}c>{\small}c>{\small}c>{\small}c>{\small}c>{\small}c}
\hline
\rowcolor{lightgray} \#tc & \#assert & transfo & node & min & max & median & mean   \\
\rowcolor{lightgray}  & & type & type & depth  & depth & depth & depth  \\ 
\hline
623 & 1041 & rep & if  &1  &3817  &1862  &1863.216 \\
\hline
\end{tabular}
\end{minipage}