
\documentclass[table]{sig-alternate}

\usepackage[T1]{fontenc} %% to get Type 1 fonts
\usepackage[utf8]{inputenc} %% to enable non ASCII characters
%\usepackage[a4paper]{geometry} %% to change the margins

\usepackage[]{mdframed}
\usepackage[]{tabularx}
\usepackage[]{graphicx}
\usepackage[]{xcolor}
\usepackage[]{pdfcomment}
\usepackage{color}
\usepackage{listings} 
\usepackage[linesnumbered]{algorithm2e}
%\usepackage{framed}
\usepackage{paralist}
\usepackage{array}
\usepackage{numprint}
%\usepackage[table]{xcolor}

%\usepackage[font=footnotesize,labelfont=bf,justification=centering]{caption}
\usepackage[justification=centering]{caption}

\newtheorem{definition}{Definition}
\lstset{ 
  basicstyle=\ttfamily\scriptsize,        
  captionpos=t,                    
  commentstyle=\color{cyan},    
  escapeinside={\%*}{*)},          
  keywordstyle=\color{blue},       
  stringstyle=\color{red},       
  numberstyle=\tiny\color{gray}, 
  stepnumber=2,                    
  title=\lstname,                   
  language=Java,
  breakatwhitespace=false,         
  breaklines=true,                 
  extendedchars=true,              
  showspaces=false,                
  showstringspaces=false,          
  showtabs=false,                  
  tabsize=2                       
} 



%\newcommand{\TODO}[1]{\textcolor{red}{#1}\pdfcomment[color=yellow,open=false]{#1}}
\newcommand{\nbprograms}{\numprint{6}\xspace}
\newcommand{\nbsosies}{XXX\xspace}
\newcommand{\TODO}[1]{\textcolor{red}{#1}}
\newcommand{\todo}{\TODO}


% ISSTA title: Tailored Source Code Transformations to Synthesize Computationally Diverse Program Variants
\title{A dataset of program sosies}
\author{}

%Hypothesis: at every transplantation point where we can delete a statement, we can also add and replace.
%Hypothesis: 
%- there are more sosies in transplantation points that are covered by a small number of test cases.
%- there are more sosies in transplantation points that are in private methods.
%Hypothesis: the most interesting sosies are at transplantation points that are covered by a large number of test cases
%In cases of delete and replace => let us check if the AST that we delete is present elsewhere in the program.
%Are all trials in “hash*” methods succesful sosies?

\begin{document}
\maketitle
\section{Introduction}

In this document we present a dataset of sosies that have been generated for large open source Java programs. 
The objective is to illustrate the diversity of situations in which we can generate good or bad sosies, as defined in our previous work \footnote{\url{https://hal.archives-ouvertes.fr/file/index/docid/938855/filename/sosies.pdf}}. 

This dataset will grow in the future and will serve as the basis to establish a taxonomy of different code regions where we can synthesize valuable program sosies. 
\section{Good sosies}
\input{rhino-hashcode.tex} %plastic computation
Listing \ref{lst:tojson} shows a sosie of  the \texttt{toJson()} method from the Google Gson library. 
The last statement of the original method is replaced by another one: instead of setting the serialization format of the \texttt{writer} it set the indent format. 
Each variant creates a JSon with slightly different formats,  and none of these formatting decisions are part of the specified domain (and actually, specifying the exact formatting of the JSon String could be considered as over-specification). 

Here, sosiefication exploits a specific kind of plasticity that we call ``code rigidities''. We have found many regions in programs where statements assign specific values to variables, while any value in a given range would be as good. Fixing one value is what we call a rigidity, and changing this value is an interesting way to create sosies that modify the program state but still deliver a correct service.

\begin{minipage}{\columnwidth}
\begin{lstlisting}[caption={\texttt{toJson} in GSON and a sosie},label={lst:tojson},language=java,numbers=left]

// Original program
void toJson(Object src, Type typeOfSrc, JsonWriter writer){
    writer.setSerializeNulls(oldSerializeNulls); } }

//sosie
void toJson(Object src, Type typeOfSrc, JsonWriter writer){
    writer.setIndent("  ") 
} }
\end{lstlisting}
\tabcolsep=0.11cm
\begin{tabular}{>{\small}c>{\small}c>{\small}c>{\small}c>{\small}c>{\small}c>{\small}c>{\small}c}
\hline
\rowcolor{lightgray} \#tc & \#assert & transfo & node & min & max & median & mean   \\
\rowcolor{lightgray}  & & type & type & depth  & depth & depth & depth  \\ 
\hline
 &  & rep &  &  &  &  & \\
\hline
\end{tabular}
\end{minipage}
 %plastic computation
\begin{minipage}{\columnwidth}
\begin{lstlisting}[caption={\texttt{writeStringToFile} in commons.io},language=java,numbers=left]
//original program
void writeStringToFile(File file, String data, Charset encoding, boolean append) throws IOException {
  OutputStream out = null;
  out = openOutputStream(file, append); 
  IOUtils.write(data, out, encoding);
  out.close(); }

// sosie  
void writeStringToFile(File file, String data, Charset encoding, boolean append) throws IOException {
  OutputStream out = null;
  out = new FileOutputStream(file, append);
  IOUtils.write(data, out, encoding);
  out.close(); }
\end{lstlisting}
\tabcolsep=0.11cm
\begin{tabular}{>{\small}c>{\small}c>{\small}c>{\small}c>{\small}c>{\small}c>{\small}c>{\small}c}
\hline
\rowcolor{lightgray} \#tc & \#assert & transfo & node & min & max & median & mean   \\
\rowcolor{lightgray}  & & type & type & depth  & depth & depth & depth  \\ 
\hline
 &  & rep &  &  &  &  & \\
\hline
\end{tabular}
\end{minipage}
 %remove check
\input{rhino-setvalidattributes.tex} %remove check
\input{rhino-stopmethod.tex} %hits a rigidity
\input{rhino-convertarg.tex} %reduces the output space
\input{decode.tex}
%CommonMath-Array2DRowRealMatrix.java:553
\begin{minipage}{\columnwidth}
\begin{lstlisting}[caption={Two variants of \texttt{getEntry} in commons.maths},language=java,numbers=left]
// Original program
public double getEntry(final int row, final int column) throws OutOfRangeException {
  MatrixUtils.checkMatrixIndex(this, row, column);
  int n = 1; //Transplant
  return data[row][column];
}

// sosie
public double getEntry(final int row, final int column) throws OutOfRangeException {
  int n = 1; //Transplant
  return data[row][column];
}
\end{lstlisting}
\tabcolsep=0.11cm
\begin{tabular}{>{\small}c>{\small}c>{\small}c>{\small}c>{\small}c>{\small}c>{\small}c>{\small}c}
\hline
\rowcolor{lightgray} \#tc & \#assert & transfo & node & min & max & median & mean   \\
\rowcolor{lightgray}  & & type & type & depth  & depth & depth & depth  \\ 
\hline
 &  & del &  &  &  &  & \\
\hline
\end{tabular}
\end{minipage}
%CommonMath-AbstractWell.java:171
\begin{minipage}{\columnwidth}
\begin{lstlisting}[caption={Two variants of \texttt{setSeed} in commons.maths},language=java,numbers=left]
// Original program
public void setSeed(final int[] seed) {
  if (seed == null) {
    setSeed(System.currentTimeMillis() + System.identityHashCode(this));
    return;
  }
  System.arraycopy(seed, 0, v, 0, FastMath.min(seed.length, v.length));
  if (seed.length < v.length) {
   for (int i = seed.length; i < v.length; ++i) {
    final long l = v[i - seed.length];
    v[i] = (int) ((1812433253l * (l ^ (l >> 30)) + i) & 0xffffffffL);
    }
  }
  index = 0;
  clear();
}
//sosie
public void setSeed(final int[] seed) {
  if (seed == null) {
    setSeed(System.currentTimeMillis() + System.identityHashCode(this));
    return;
  }
  System.arraycopy(seed, 0, v, 0, FastMath.min(seed.length, v.length));
  if (seed.length < v.length) {
   for (int i = seed.length; i < v.length; ++i) {
    final long l = v[i - seed.length];
    v[i] = (int) ((1812433253l * (l ^ (l >> 30)) + i) & 0xffffffffL);
    }
  }
  index = 0;

}
\end{lstlisting}
\tabcolsep=0.11cm
\begin{tabular}{>{\small}c>{\small}c>{\small}c>{\small}c>{\small}c>{\small}c>{\small}c>{\small}c}
\hline
\rowcolor{lightgray} \#tc & \#assert & transfo & node & min & max & median & mean   \\
\rowcolor{lightgray}  & & type & type & depth  & depth & depth & depth  \\ 
\hline
&  & del &  &  &  &  & \\
\hline
\end{tabular}
\end{minipage}
\input{invoke.tex}
\input{rhino-getcharignorelineend.tex}
\section{Useless sosies}
\label{sec:useless}
\input{capture-tostring.tex} %adds redundant code
\input{rhino-ensureIndex.tex}
\section{To-be-discussed sosies}
%EasyMock 3.2-BridgeMethodResolver:254
\begin{minipage}{\columnwidth}
\begin{lstlisting}[caption={\texttt{createTypeVariableMap and a sosie} in EasyMock},language=java,numbers=left]
// Original program
private static Map<TypeVariable<?>, Type> createTypeVariableMap(final Class<?> cls) {
  final Map<TypeVariable<?>, Type> typeVariableMap = new HashMap<TypeVariable<?>, Type>();
  extractTypeVariablesFromGenericInterfaces(cls.getGenericInterfaces(), typeVariableMap);
  Type genericType = cls.getGenericSuperclass();
  Class<?> type = cls.getSuperclass();
  while (!Object.class.equals(type)) {
    if (genericType instanceof ParameterizedType) {
      final ParameterizedType pt = (ParameterizedType) genericType;
      populateTypeMapFromParameterizedType(pt, typeVariableMap);
      }
      extractTypeVariablesFromGenericInterfaces(type.getGenericInterfaces(), typeVariableMap);
      genericType = type.getGenericSuperclass();
      type = type.getSuperclass();
    }
    type = cls;
    while (type.isMemberClass()) {
      genericType = type.getGenericSuperclass();
      if (genericType instanceof ParameterizedType) {
        final ParameterizedType pt = (ParameterizedType) genericType;
        populateTypeMapFromParameterizedType(pt, typeVariableMap);
      }
      type = type.getEnclosingClass();
    }                 
  return typeVariableMap;
}
//sosie
private static Map<TypeVariable<?>, Type> createTypeVariableMap(final Class<?> cls) {
  final Map<TypeVariable<?>, Type> typeVariableMap = new HashMap<TypeVariable<?>, Type>();
  extractTypeVariablesFromGenericInterfaces(cls.getGenericInterfaces(), typeVariableMap);
  Type genericType = cls.getGenericSuperclass();
  Class<?> type = cls.getSuperclass();
  while (!Object.class.equals(type)) {
    if (genericType instanceof ParameterizedType) {
      final ParameterizedType pt = (ParameterizedType) genericType;
      populateTypeMapFromParameterizedType(pt, typeVariableMap);
      }
      extractTypeVariablesFromGenericInterfaces(type.getGenericInterfaces(), typeVariableMap);
      genericType = type.getGenericSuperclass();
      type = type.getSuperclass();
    }
    type = cls;
  return typeVariableMap;
}
\end{lstlisting}
\tabcolsep=0.11cm
\begin{tabular}{>{\small}c>{\small}c>{\small}c>{\small}c>{\small}c>{\small}c>{\small}c>{\small}c}
\hline
\rowcolor{lightgray} \#tc & \#assert & transfo & node & min & max & median & mean   \\
\rowcolor{lightgray}  & & type & type & depth  & depth & depth & depth  \\ 
\hline
&  & del &  &  &  &  & \\
\hline
\end{tabular}
\end{minipage}
%easymock.internal.matchers.equals:39
\begin{minipage}{\columnwidth}
\begin{lstlisting}[caption={Two variants of \texttt{matches} in EasyMock},language=java,numbers=left]
//original
public boolean matches(final Object actual) {
  if (this.expected == null) {
    return actual == null;
  }
  return expected.equals(actual);
}
//sosie
public boolean matches(final Object actual) {
  if (this.expected == null) {
    return true;
  }
  return expected.equals(actual);
}
\end{lstlisting}
\tabcolsep=0.11cm
\begin{tabular}{>{\small}c>{\small}c>{\small}c>{\small}c>{\small}c>{\small}c>{\small}c>{\small}c}
\hline
\rowcolor{lightgray} \#tc & \#assert & transfo & node & min & max & median & mean   \\
\rowcolor{lightgray}  & & type & type & depth  & depth & depth & depth  \\ 
\hline
&  & rep &  &  &  &  & \\
\hline
\end{tabular}
\end{minipage}
\input{isthreadsafe.tex}
\input{rhino-setlp.tex}
\section{Bad sosies}


% to add the bibliography, uncomment the following lines
% \bibliography{references}
% \bibliographystyle{plain}
\end{document}
