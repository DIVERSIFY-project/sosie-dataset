%\documentclass{sig-alternate}
\documentclass{article}
\usepackage[T1]{fontenc} 
\usepackage[utf8]{inputenc} 

\usepackage[]{mdframed}
\usepackage[]{tabularx}
\usepackage[]{graphicx}
\usepackage[table]{xcolor}
\usepackage[]{pdfcomment}
\usepackage{listings} 
\usepackage[linesnumbered]{algorithm2e}
\usepackage{framed}
\usepackage{paralist}
\usepackage{numprint}
\usepackage{color}
%\usepackage{ulem}
\usepackage{array}
\usepackage[justification=centering]{caption}
\usepackage[inline]{enumitem}
% \usepackage{tikz}
% \newcommand\bh{\tikz[remember picture]
%                   \node (begin highlight) {};
%                }
% \newcommand\eh{\tikz[remember picture]
%                \node (end highlight) {};
%                \tikz[remember picture, overlay] 
%                \draw[black,thin] (begin highlight) -- (end
%     %          \draw[yellow,line width=10pt,opacity=0.3] (begin highlight) -- (end highlight);
%                 highlight);
%                }

\newtheorem{definition}{Definition}
\lstset{ 
  language=Java,
  basicstyle=\ttfamily\scriptsize,   
  moredelim=[is][\color{gray}\ttfamily]{_}{_},
  moredelim=[is][\color{blue}\underbar]{(*}{*)},
  captionpos=t,                    
  commentstyle=\color{cyan},    
  escapeinside={\%*}{*)},          
  keywordstyle=\color{blue},       
  stringstyle=\color{red},       
  numberstyle=\tiny\color{gray}, 
  stepnumber=2,                    
  title=\lstname,                   
  breakatwhitespace=false,         
  breaklines=true,                 
  extendedchars=true,              
  showspaces=false,                
  showstringspaces=false,          
  showtabs=false,                  
  tabsize=2,
} 



%\newcommand{\TODO}[1]{\textcolor{red}{#1}\pdfcomment[color=yellow,open=false]{#1}}
\newcommand{\nbprograms}{\numprint{6}\xspace}
\newcommand{\nbsosies}{XXX\xspace}
\newcommand{\TODO}[1]{\textcolor{red}{#1}}
\newcommand{\todo}{\TODO}


% ISSTA title: Tailored Source Code Transformations to Synthesize Computationally Diverse Program Variants
\title{A dataset of program sosies}
\author{}

%Hypothesis: at every transplantation point where we can delete a statement, we can also add and replace.
%Hypothesis: 
%- there are more sosies in transplantation points that are covered by a small number of test cases.
%- there are more sosies in transplantation points that are in private methods.
%Hypothesis: the most interesting sosies are at transplantation points that are covered by a large number of test cases
%In cases of delete and replace => let us check if the AST that we delete is present elsewhere in the program.
%Are all trials in “hash*” methods succesful sosies?

\begin{document}
\maketitle
\section{Introduction}

In this document we present a dataset of sosies that have been generated for large open source Java programs. 
The objective is to illustrate the diversity of situations in which we can generate good or bad sosies, as defined in our previous work \footnote{\url{https://hal.archives-ouvertes.fr/file/index/docid/938855/filename/sosies.pdf}}. 

This dataset will grow in the future and will serve as the basis to establish a taxonomy of different code regions where we can synthesize valuable program sosies. 
\section{Good sosies}
Some program regions implement behavior which specfication is intrinsically plastic, i.e., behaviors for which the correctness of the output is not binary (one single possible correct value, all others are wrong) but has to be in a range of values. 
These regions are extremely forgiving and provide great opportunities for sosiefication since it can transform the programs in many ways and still produce valuable functionality, as long as the outcome remains in a range. 

One situation that we have encountered many times relates to the production of hash keys.
Methods that produce these keys have a very plastic specification: they must return an integer value that can be used to identify an element. There is no constraint on the value of the key, but there is one constraint on the behavior of the hash function: it must be deterministic.
Listing \ref{lst:hashcode} is an example of such a sosie.

% position: org.mozilla.classfile.FieldOrMethodRef:4776
\begin{minipage}{\columnwidth}
\begin{lstlisting}[caption={\texttt{hashCode} in Rhino and a sosie},label={lst:hashcode},language=java,numbers=left]
//original
public int hashCode(){}
  if (hashCode == -1) {
  int h1 = className.hashCode();
  int h2 = name.hashCode();
  int h3 = type.hashCode();
  hashCode = h1 ^ h2 ^ h3;
  }
  return hashCode;
}
//sosie
public int hashCode(){}
  if (hashCode == -1) {
  int h1 = className.length();
  int h2 = name.hashCode();
  int h3 = type.hashCode();
  hashCode = h1 ^ h2 ^ h3;
  }
  return hashCode;
}
\end{lstlisting}
\tabcolsep=0.11cm
\begin{tabular}{>{\small}c>{\small}c>{\small}c>{\small}c>{\small}c>{\small}c>{\small}c>{\small}c}
\hline
\rowcolor{lightgray} \#tc & \#assert & transfo & node & min & max & median & mean   \\
\rowcolor{lightgray}  & & type & type & depth  & depth & depth & depth  \\ 
\hline
&  & rep &  &  &  &  & \\
\hline
\end{tabular}
\end{minipage} %plastic computation
\input{sosie016-gson-tojson.tex} %plastic computation
In listing \ref{lst:writeStringToFile}, the original program calls \texttt{openOutputStream}, which checks different things about the file name, while the sosie  directly calls the constructor of \texttt{FileOutputStream}. 
In all nominal cases, these two programs behave exactly in the same way, and the sosie executes less code. 
In exceptional cases, i.e. when \texttt{writeStringToFile()} is called with an invalid file name, the original program handles it, while the variant throws a \texttt{FileNotFoundException}. 
The original program and the sosie exhibit failure diversity on exceptional cases.

Considering the parts of the program that handle checks and exceptional cases as plasticity in the specification and the implementation is conceptually very close to the ideas of exploration of the correctness envelop as expressed by Rinard and colleagues \footnote{http://people.csail.mit.edu/rinard/paper/oopsla05.pdf}.


%org.apache.commons.io.FileUtils
\begin{minipage}{\columnwidth}
\begin{lstlisting}[caption={\texttt{writeStringToFile} in commons.io},label={lst:writeStringToFile},language=java,numbers=left]
//original program
void writeStringToFile(File file, String data, Charset encoding, boolean append) throws IOException {
  OutputStream out = null;
  out = openOutputStream(file, append); 
  IOUtils.write(data, out, encoding);
  out.close(); }

// sosie  
void writeStringToFile(File file, String data, Charset encoding, boolean append) throws IOException {
  OutputStream out = null;
  out = new FileOutputStream(file, append); //this statement replaces the original
  IOUtils.write(data, out, encoding);
  out.close(); }
\end{lstlisting}
\tabcolsep=0.11cm
\begin{tabular}{>{\small}c>{\small}c>{\small}c>{\small}c>{\small}c>{\small}c>{\small}c>{\small}c}
\hline
\rowcolor{lightgray} \#tc & \#assert & transfo & node & min & max & median & mean   \\
\rowcolor{lightgray}  & & type & type & depth  & depth & depth & depth  \\ 
\hline
 &  & rep &  &  &  &  & \\
\hline
\end{tabular}
\end{minipage}
 %remove check
Listing \ref{lst:setAttributes} is an example of sosie that removes checks on inputs.

%position: org.mozilla.javascript.IdScriptableObject:431
\begin{minipage}{\columnwidth}
\begin{lstlisting}[caption={\texttt{setAttributes} in Rhino and a sosie},label=lst:setAttributes,language=java,numbers=left]
//original
public void setAttributes(String name, int attributes){
  ScriptableObject.checkValidAttributes(attributes);
  %* {\color{gray} \emph{...} } *)
}
//sosie
public void setAttributes(String name, int attributes){
  while (attributes < attributes) {
    ++attributes;
    attributes <<= 1;
  }
  %* {\color{gray} \emph{...} } *)
}\end{lstlisting}
\tabcolsep=0.11cm
\begin{tabular}{>{\small}c>{\small}c>{\small}c>{\small}c>{\small}c>{\small}c>{\small}c>{\small}c}
\hline
\rowcolor{lightgray} \#tc & \#assert & transfo & node & min & max & median & mean   \\
\rowcolor{lightgray}  & & type & type & depth  & depth & depth & depth  \\ 
\hline
&  & rep &  &  &  &  & \\
\hline
\end{tabular}
\end{minipage}


 %remove check
Listing \ref{lst:stopmethod} is an example of sosie that hits a rigidity: assigning 0 or 4 to \texttt{itsMaxStack} has no incidence on the behavior of Rhino.


%position: org.mozilla.classfile.ClassFileWriter:438
\begin{minipage}{\columnwidth}
\begin{lstlisting}[caption={\texttt{stopMethod} in Rhino and a sosie},label={lst:stopmethod},language=java,numbers=left]
//original
public void stopMethod(short maxLocals) {
  %* {\color{gray} \emph{...} } *)
  itsMaxStack = 0;
  itsStackTop = 0;
  itsLabelTableTop = 0;
  itsFixupTableTop = 0;
  itsVarDescriptors = null;
  itsSuperBlockStarts = null;
  itsSuperBlockStartsTop = 0;
  itsJumpFroms = null;
}
//sosie
public void stopMethod(short maxLocals) {
  %* {\color{gray} \emph{...} } *)
  itsMaxStack = 4;
  itsStackTop = 0;
  itsLabelTableTop = 0;
  itsFixupTableTop = 0;
  itsVarDescriptors = null;
  itsSuperBlockStarts = null;
  itsSuperBlockStartsTop = 0;
  itsJumpFroms = null;
}
%* {\color{gray} \emph{...} } *)
\end{lstlisting}
\tabcolsep=0.11cm
\begin{tabular}{>{\small}c>{\small}c>{\small}c>{\small}c>{\small}c>{\small}c>{\small}c>{\small}c}
\hline
\rowcolor{lightgray} \#tc & \#assert & transfo & node & min & max & median & mean   \\
\rowcolor{lightgray}  & & type & type & depth  & depth & depth & depth  \\ 
\hline
&  & rep &  &  &  &  & \\
\hline
\end{tabular}
\end{minipage} %hits a rigidity
Listing \ref{lst:convertarg} is a beautiful sosie, which reduces the output space of the program, increasing safety.

%position: rg.mozilla.javascript.FunctionObject:205
\begin{minipage}{\columnwidth}
\begin{lstlisting}[caption={\texttt{convertArg} in Rhino and a sosie},label={lst:convertarg},language=java,numbers=left]
//original
public static Object convertArg(Context cx, Scriptable scope, Object arg, int typeTag){
  switch (typeTag) {
    %* {\color{gray} \emph{...} } *)
    case JAVA_OBJECT_TYPE:
      return arg;
    default:
        throw new IllegalArgumentException();
  }
}
//sosie
public static Object convertArg(Context cx, Scriptable scope, Object arg, int typeTag){
  switch (typeTag) {
    %* {\color{gray} \emph{...} } *)
    case JAVA_OBJECT_TYPE:
      if (arg != null){return arg;}
    default:
      throw new IllegalArgumentException();
  }
}

\end{lstlisting}
\tabcolsep=0.11cm
\begin{tabular}{>{\small}c>{\small}c>{\small}c>{\small}c>{\small}c>{\small}c>{\small}c>{\small}c}
\hline
\rowcolor{lightgray} \#tc & \#assert & transfo & node & min & max & median & mean   \\
\rowcolor{lightgray}  & & type & type & depth  & depth & depth & depth  \\ 
\hline
&  & rep &  &  &  &  & \\
\hline
\end{tabular}
\end{minipage} %reduces the output space
Listing \ref{lst:center} is an example of sosie that exploits redundancy in the code. The statement \texttt{if (isEmpty(padStr)) { 
  padStr = SPACE;
} } at line \ref{l:isempty} assigns a value to the \texttt{padStr} variable, then this variable is passed when calling methods \texttt{leftPad} and \texttt{rightPad}. Yet, each of these two methods include the exact same statement, which will eventually assign a value to \texttt{padStr}. So, the statement is redundant and can be removed from the sosie.


%org.apache.commons.lang.StringUtils:5553
%source: "redundant checking.txt" in the dropbox folder
\begin{minipage}{\columnwidth}
\begin{lstlisting}[caption={\texttt{center} in commons.lang and a sosie},label={lst:center},language=java,numbers=left]
//original
public static String center(String str, final int size, String padStr) {
  if (str == null || size <= 0) {return str;}
  if (isEmpty(padStr)) { %*\label{l:isempty}*)
    padStr = SPACE;
  } 
  final int strLen = str.length();
  final int pads = size - strLen;
  if (pads <= 0) {return str;}
  str = leftPad(str, strLen + pads / 2, padStr);
  str = rightPad(str, size, padStr);
  return str;}
//sosie
public static String center(String str, final int size, String padStr) {
  if (str == null || size <= 0) {return str;}
  %*{\color{orange} \textbf{if-stmt deleted} }*)
  final int strLen = str.length();
  final int pads = size - strLen;
  if (pads <= 0) {return str;}
  str = leftPad(str, strLen + pads / 2, padStr);
  str = rightPad(str, size, padStr);
  return str;}
\end{lstlisting}
\tabcolsep=0.11cm
\begin{tabular}{>{\small}c>{\small}c>{\small}c>{\small}c>{\small}c>{\small}c>{\small}c>{\small}c}
\hline
\rowcolor{lightgray} \#tc & \#assert & transfo & node & min & max & median & mean   \\
\rowcolor{lightgray}  & & type & type & depth  & depth & depth & depth  \\ 
\hline
&  & del &  &  &  &  & \\
\hline
\end{tabular}
\end{minipage} %removes redundant computation
\input{sosie001-gson-canonicalize.tex} %remove a complete functionality 
Listing \ref{lst:range-tostring} shows a sosie for the \texttt{toString()} method in the \texttt{Range} class of commons.lang. 
This method builds a String value and saves it in the \texttt{toString} attribute for future usage (a sort of cache to save computation in the future). 
The sosie removes this cache operation thus reducing a bit the performance while maintaining the same service.

%org.apache.commons.lang.Range:458
%source:"invalidating a cache mechanism.txt" in dropbox
\begin{minipage}{\columnwidth}
\begin{lstlisting}[caption={\texttt{toString} in commons.lang and a sosie},label={lst:range-tostring},language=java,numbers=left]
//original
public String toString() {
  String result = toString;
  if (result == null) {
    final StringBuilder buf = new StringBuilder(32);
    buf.append('[');
    buf.append(minimum);
    buf.append("..");
    buf.append(maximum);
    buf.append(']');
    result = buf.toString();
    toString = result;
  }
  return result;}
//sosie
public String toString() {
  String result = toString;
  if (result == null) {
    final StringBuilder buf = new StringBuilder(32);
    buf.append('[');
    buf.append(minimum);
    buf.append("..");
    buf.append(maximum);
    buf.append(']');
    result = buf.toString();
    %*{\textbf{\color{orange}assignment deleted}}*)
  }
  return result;}
\end{lstlisting}
\tabcolsep=0.11cm
\begin{tabular}{>{\small}c>{\small}c>{\small}c>{\small}c>{\small}c>{\small}c>{\small}c>{\small}c}
\hline
\rowcolor{lightgray} \#tc & \#assert & transfo & node & min & max & median & mean   \\
\rowcolor{lightgray}  & & type & type & depth  & depth & depth & depth  \\ 
\hline
2&  & del &stmt list  &1  & 2 & 2 & 1.5\\
\hline
\end{tabular}
\end{minipage} %remove a cache
Listing \ref{lst:range-hashCode} is an example where sosiefication exploits plasticity in the computation, which can be found in many programs. 
The \texttt{hashCode()} method must return an integer value that can be used to quickly retrieve a value in a collection. Yet, the exact value of this integer is not part of the specification, i.e., there are many ways to compute this value. 
Thus, removing a statement in this method does not change the validity of the service provided by the function. 

%org.apache.commons.lang.Range:433
%source:"plastic numeric function.txt" in dropbox
\begin{minipage}{\columnwidth}
\begin{lstlisting}[caption={\texttt{hashCode} in commons.lang and a sosie},label={lst:range-hashCode},language=java,numbers=left]
//original
public int hashCode() {
  int result = hashCode;
  if (hashCode == 0) {
    result = 17;
    result = 37 * result + getClass().hashCode();
    result = 37 * result + minimum.hashCode(); 
    result = 37 * result + maximum.hashCode();
    hashCode = result;
  }
  return result;}
//sosie
public int hashCode() {
  int result = hashCode;
  if (hashCode == 0) {
    result = 17;
    result = 37 * result + getClass().hashCode();
    %*{\textbf{\color{orange}assignment deleted}}*)
    result = 37 * result + maximum.hashCode();
    hashCode = result;
  }
  return result;}
\end{lstlisting}
\tabcolsep=0.11cm
\begin{tabular}{>{\small}c>{\small}c>{\small}c>{\small}c>{\small}c>{\small}c>{\small}c>{\small}c}
\hline
\rowcolor{lightgray} \#tc & \#assert & transfo & node & min & max & median & mean   \\
\rowcolor{lightgray}  & & type & type & depth  & depth & depth & depth  \\ 
\hline
&  & del &  &  &  &  & \\
\hline
\end{tabular}
\end{minipage} %modify hashcode function
Listing \ref{lst:nextLong} displays an excerpt of the \texttt{nextLong()} method in GSon. 
Under certain conditions, the method returns a value, skipping all the computation performed after the \texttt{try} block. 
Yet, this is a shortcut in the computation, i.e., if we remove the \texttt{return} statement, the subsequent code will return exactly the same value, with some additional computation. 

%com.google.json.stream.JsonReader:957
%source:"Sosies-GSON-21-08-2015.txt" in dropbox
\begin{minipage}{\columnwidth}
\begin{lstlisting}[caption={\texttt{nextLong} in GSon and a sosie},label={lst:nextLong},language=java,numbers=left]
//original
public long nextLong() throws IOException {
  *{\color{grey}...}*)
  if (p == PEEKED_NUMBER) {
    peekedString = new String(buffer, pos, peekedNumberLength);
    pos += peekedNumberLength;
  } else if (p == PEEKED_SINGLE_QUOTED || p == PEEKED_DOUBLE_QUOTED) {
    peekedString = nextQuotedValue(p == PEEKED_SINGLE_QUOTED ? '\'' : '"');
    try {
      long result = Long.parseLong(peekedString);
      peeked = PEEKED_NONE;
      pathIndices[stackSize - 1]++;
      return result; 
      } 
  *{\color{grey}...}*)
  return result;}
//sosie
public long nextLong() throws IOException {
  *{\color{grey}...}*)
  if (p == PEEKED_NUMBER) {
    peekedString = new String(buffer, pos, peekedNumberLength);
    pos += peekedNumberLength;
  } else if (p == PEEKED_SINGLE_QUOTED || p == PEEKED_DOUBLE_QUOTED) {
    peekedString = nextQuotedValue(p == PEEKED_SINGLE_QUOTED ? '\'' : '"');
    try {
      long result = Long.parseLong(peekedString);
      peeked = PEEKED_NONE;
      pathIndices[stackSize - 1]++;
      *{\textbf{\color{orange}assignment deleted}}*)
      } 
  *{\color{grey}...}*)
  return result;}
\end{lstlisting}
\tabcolsep=0.11cm
\begin{tabular}{>{\small}c>{\small}c>{\small}c>{\small}c>{\small}c>{\small}c>{\small}c>{\small}c}
\hline
\rowcolor{lightgray} \#tc & \#assert & transfo & node & min & max & median & mean   \\
\rowcolor{lightgray}  & & type & type & depth  & depth & depth & depth  \\ 
\hline
&  & del &  &  &  &  & \\
\hline
\end{tabular}
\end{minipage}
%*{\textbf{\color{orange}assignment deleted}}*)
%*{\color{grey}...}*) %removes a shortcut in the computation
\begin{minipage}{\columnwidth}
\begin{lstlisting}[caption={\texttt{decode} in commons.codec and a sosie},language=java,numbers=left]
// Original program
void decode(final byte[] in, int inPos, final int inAvail, final Context context) {
  switch (context.modulus) {
    case 0 : // impossible, as excluded above
    case 1 : // 6 bits - ignore entirely
        // not currently tested; perhaps it is impossible?
             break;
}

// sosie
void decode(final byte[] in, int inPos, final int inAvail, final Context context) {
  switch (context.modulus) {
    case 0 : // impossible, as excluded above
    case 1 : 
}
\end{lstlisting}
\tabcolsep=0.11cm
\begin{tabular}{>{\small}c>{\small}c>{\small}c>{\small}c>{\small}c>{\small}c>{\small}c>{\small}c}
\hline
\rowcolor{lightgray} \#tc & \#assert & transfo & node & min & max & median & mean   \\
\rowcolor{lightgray}  & & type & type & depth  & depth & depth & depth  \\ 
\hline
 &  & del &  &  &  &  & \\
\hline
\end{tabular}
\end{minipage}
\input{sosie005-maths-getentry.tex}
\input{sosie015-maths-setseed.tex}
%EasyMock 3.2-ReplayState.java:38
\begin{minipage}{\columnwidth}
\begin{lstlisting}[caption={Two variants of \texttt{invoke} in EasyMock},language=java,numbers=left]
//original
public Object invoke(final Invocation invocation) throws Throwable {
  behavior.checkThreadSafety();
  if (behavior.isThreadSafe()) {
    lock.lock();
    try {
      return invokeInner(invocation);
    } finally {
      lock.unlock();
    }
  }
  return invokeInner(invocation);
}
//sosie
public Object invoke(final Invocation invocation) throws Throwable {
  behavior.checkThreadSafety();
  if (behavior.isThreadSafe()) {
    new locks.ReentrantLock();
  }
  return invokeInner(invocation);
}
\end{lstlisting}
\tabcolsep=0.11cm
\begin{tabular}{>{\small}c>{\small}c>{\small}c>{\small}c>{\small}c>{\small}c>{\small}c>{\small}c}
\hline
\rowcolor{lightgray} \#tc & \#assert & transfo & node & min & max & median & mean   \\
\rowcolor{lightgray}  & & type & type & depth  & depth & depth & depth  \\ 
\hline
&  & rep &  &  &  &  & \\
\hline
\end{tabular}
\end{minipage}
In listing \ref{lst:getCharIgnoreLineEnd}, the variable \texttt{c} is never equal to  `r' (maybe different on windows) 

%position: org.mozilla.javascript.TokenStream:1359
\begin{minipage}{\columnwidth}
\begin{lstlisting}[caption={\texttt{getCharIgnoreLineEnd} in Rhino and a sosie},label={lst:getCharIgnoreLineEnd},language=java,numbers=left]
//original
private int getCharIgnoreLineEnd() throws IOException{
  %* {\color{gray} \emph{...} } *)
  if (c <= 127) {
    if (c == '\n' || c == '\r') {
      lineEndChar = c;
      c = '\n';
    }
  } else {
  %* {\color{gray} \emph{...} } *)
}
private int getCharIgnoreLineEnd() throws IOException{
  %* {\color{gray} \emph{...} } *)
  if (c <= 127) {
    if (c == '\n' || c == '\r') {
      lineEndChar = c;
      c = (c) > c ? c : c;
      }
  } else {
  %* {\color{gray} \emph{...} } *)
}
\end{lstlisting}
\tabcolsep=0.11cm
\begin{tabular}{>{\small}c>{\small}c>{\small}c>{\small}c>{\small}c>{\small}c>{\small}c>{\small}c}
\hline
\rowcolor{lightgray} \#tc & \#assert & transfo & node & min & max & median & mean   \\
\rowcolor{lightgray}  & & type & type & depth  & depth & depth & depth  \\ 
\hline
&  & rep &  &  &  &  & \\
\hline
\end{tabular}
\end{minipage}



%org.apache.commons.collections4.map.AbstractHashedMap:396
\begin{figure}[ht]
\begin{lstlisting}[caption={\texttt{hash} in commons.collection},label={lst:coll-hash},language=java,numbers=left]
//original program
int hash(final Object key) {
    int h = key.hashCode();
    h += ~(h << 9); 
    h ^=  h >>> 14;
    h +=  h << 4;
    h ^=  h >>> 10;
    return h;}
// sosie  
int hash(final Object key) {
  int h = key.hashCode();
  %*{\textbf{\color{orange}the assignment is deleted or the following  statements are added}}*)
  %*{\textbf{\color{orange}h+= ~(h<< 11); }}*)
  %*{\textbf{\color{orange}this.h)++; }}*)
  %*{\textbf{\color{orange}h= (h* 37) + (h); }}*)
  %*{\textbf{\color{orange}h= --(h); }}*)
  %*{\textbf{\color{orange}h= h; }}*)
  %*{\textbf{\color{orange}size -= nCopies; }}*)
  %*{\textbf{\color{orange}h += h << 4; }}*)
  h ^=  h >>> 14;
  h +=  h << 4;
  h ^=  h >>> 10;
  return h;}
\end{lstlisting}
\tabcolsep=0.11cm
\begin{tabular}{>{\small}c>{\small}c>{\small}c>{\small}c>{\small}c>{\small}c>{\small}c>{\small}c}
\hline
\rowcolor{lightgray} \#tc & \#assert & transfo & node & min & max & median & mean   \\
\rowcolor{lightgray}  & & type & type & depth  & depth & depth & depth  \\ 
\hline
 &  & del,add &  &  &  &  & \\
\hline
\end{tabular}
\end{figure}

%*{\textbf{\color{orange}hash ^= hash >>> 6; }}*)
%*{\textbf{\color{orange}h ^= h >>> 10;}}*) 

\section{Fooler sosies}
\label{sec:foolers}
%EasyMock 3.2-Capture:131
\begin{minipage}{\columnwidth}
\begin{lstlisting}[caption={\texttt{toString} in EasyMock and a sosie},language=java,numbers=left]
//original
public String toString() {
  if (values.isEmpty()) {
    return "Nothing captured yet";
  }
  if (values.size() == 1) {
    return String.valueOf(values.get(0));
  }
  return values.toString();
}
//sosie
public String toString() {
  if (values.isEmpty()) {
    return "Nothing captured yet";
  }
  if (values.size() == 1) {
    if ((values.size()) == 1) {
      return String.valueOf(values.get(0));
    }
  }
  return values.toString();
}
\end{lstlisting}
\tabcolsep=0.11cm
\begin{tabular}{>{\small}c>{\small}c>{\small}c>{\small}c>{\small}c>{\small}c>{\small}c>{\small}c}
\hline
\rowcolor{lightgray} \#tc & \#assert & transfo & node & min & max & median & mean   \\
\rowcolor{lightgray}  & & type & type & depth  & depth & depth & depth  \\ 
\hline
&  & rep &  &  &  &  & \\
\hline
\end{tabular}
\end{minipage} %adds redundant code
Listing \ref{lst:ensureIndex} is a valid sosie, not beautiful: the first assignment of \texttt{index} is useless

%Position: org.mozilla.javascript.UintMap:290
\begin{minipage}{\columnwidth}
\begin{lstlisting}[caption={\texttt{ensureIndex} in Rhino and a sosie},label=lst:ensureIndex,language=java,numbers=left]
//original
private int ensureIndex(int key, boolean intType) {
  int index = -1;
  //end transformation
  int firstDeleted = -1;
  int[] keys = this.keys;
  if (keys != null) {
    int fraction = key * A;
    index = fraction >>> (32 - power);
    %* {\color{gray} \emph{...} } *)
    }
    // Inserting of new key
    if (check && keys != null && keys[index] != EMPTY)
        Kit.codeBug();
    if (firstDeleted >= 0) {
        index = firstDeleted;
    }
    else {
      // Need to consume empty entry: check occupation level
      if (keys == null || occupiedCount * 4 >= (1 << power) * 3) {
        // Too litle unused entries: rehash
        rehashTable(intType);
        return insertNewKey(key);
      }
    ++occupiedCount;
  }
  keys[index] = key;
  ++keyCount;
  return index;
}
//sosie
private int ensureIndex(int key, boolean intType) {
  int index = 65536;
  //end transformation
  int firstDeleted = -1;
  int[] keys = this.keys;
  if (keys != null) {
    int fraction = key * A;
    index = fraction >>> (32 - power);
    %* {\color{gray} \emph{...} } *)
    }
    // Inserting of new key
    if (check && keys != null && keys[index] != EMPTY)
        Kit.codeBug();
    if (firstDeleted >= 0) {
        index = firstDeleted;
    }
    else {
      // Need to consume empty entry: check occupation level
      if (keys == null || occupiedCount * 4 >= (1 << power) * 3) {
        // Too litle unused entries: rehash
        rehashTable(intType);
        return insertNewKey(key);
      }
    ++occupiedCount;
  }
  keys[index] = key;
  ++keyCount;
  return index;
}
\end{lstlisting}
\tabcolsep=0.11cm
\begin{tabular}{>{\small}c>{\small}c>{\small}c>{\small}c>{\small}c>{\small}c>{\small}c>{\small}c}
\hline
\rowcolor{lightgray} \#tc & \#assert & transfo & node & min & max & median & mean   \\
\rowcolor{lightgray}  & & type & type & depth  & depth & depth & depth  \\ 
\hline
&  & rep &  &  &  &  & \\
\hline
\end{tabular}
\end{minipage}
Listing \ref{lst:ensureCapacity} is an example of a fooler sosie, which adds a recursive call to \texttt{ensureCapacity()}. 
This could turn the method into an infinite recursion, except that in the additional recursive call, the value of the parameter is always such that the condition of the first if-statement  always holds true and thus the method execution immediately stop. 
The additional call adds a harmless method call in the execution flow.

%org.apache.commons.collections4.map.AbstractHashedMap:628
\begin{figure}[ht]
\begin{lstlisting}[caption={\texttt{ensureCapacity} in commons.collection},label={lst:ensureCapacity},language=java,numbers=left]
//original program
protected void ensureCapacity(final int newCapacity) {
  final int oldCapacity = data.length;
  if (newCapacity <= oldCapacity) {
    return;
  }
  if (size == 0) {
    threshold = calculateThreshold(newCapacity, loadFactor);
    data = new HashEntry[newCapacity];
  } else {
    final HashEntry<K, V> oldEntries[] = data;
    final HashEntry<K, V> newEntries[] = new HashEntry[newCapacity];
    modCount++;
    for (int i = oldCapacity - 1; i >= 0; i--) {
      HashEntry<K, V> entry = oldEntries[i];
        if (entry != null) {
          oldEntries[i] = null;  // gc
          do {
            final HashEntry<K, V> next = entry.next;
              final int index = hashIndex(entry.hashCode, newCapacity);
              entry.next = newEntries[index];
              newEntries[index] = entry;
              entry = next;
          } while (entry != null);
        }
      }
      threshold = calculateThreshold(newCapacity, loadFactor);
      data = newEntries;
}}
// sosie  
protected void ensureCapacity(final int newCapacity) {
  final int oldCapacity = data.length;
  if (newCapacity <= oldCapacity) {
    return;
  }
  if (size == 0) {
    threshold = calculateThreshold(newCapacity, loadFactor);
    data = new HashEntry[newCapacity];
  } else {
    final HashEntry<K, V> oldEntries[] = data;
    final HashEntry<K, V> newEntries[] = new HashEntry[newCapacity];
    modCount++;
    for (int i = oldCapacity - 1; i >= 0; i--) {
      HashEntry<K, V> entry = oldEntries[i];
        if (entry != null) {
          oldEntries[i] = null;  // gc
          do {
            final HashEntry<K, V> next = entry.next;
              final int index = hashIndex(entry.hashCode, newCapacity);
              entry.next = newEntries[index];
              newEntries[index] = entry;
              entry = next;
          } while (entry != null);
        }
      }
      threshold = calculateThreshold(newCapacity, loadFactor);
      data = newEntries;
    }
    %*{\textbf{\color{orange}following statement added}}*)
    ensureCapacity(threshold)}
\end{lstlisting}
\tabcolsep=0.11cm
\begin{tabular}{>{\small}c>{\small}c>{\small}c>{\small}c>{\small}c>{\small}c>{\small}c>{\small}c}
\hline
\rowcolor{lightgray} \#tc & \#assert & transfo & node & min & max & median & mean   \\
\rowcolor{lightgray}  & & type & type & depth  & depth & depth & depth  \\ 
\hline
 &  & add &  &  &  &  & \\
\hline
\end{tabular}
\end{figure}
Listing \ref{lst:utils-get} illustrates two kinds of plasticity exploited by the sosie. 
First, it exploits the implementation redundancy inside the program, i.e., \texttt{((Object[]) object)[i]} and \texttt{Array.get(object, i)} have the same behavior, yet are based on completely different implementations. 
Second, it also introduced extra checking code, as well as failure diversity.

%org.apache.commons.collections4.CollectionUtils:1260
\begin{figure}[ht]
\begin{lstlisting}[caption={\texttt{get} in commons.collection},label={lst:utils-get},language=java,numbers=left]
//original program
public static Object get(final Object object, final int index) {
  int i = index;
  if (i < 0) {
    throw new IndexOutOfBoundsException("Index cannot be negative: " + i);
  }
  if (object instanceof Map<?,?>) {
    final Map<?, ?> map = (Map<?, ?>) object;
    final Iterator<?> iterator = map.entrySet().iterator();
    return get(iterator, i);
  } else if (object instanceof Object[]) {
    return ((Object[]) object)[i];
    } else if (object instanceof Iterator<?>) {
      final Iterator<?> it = (Iterator<?>) object;
      while (it.hasNext()) {
        i--;
        if (i == -1) {return it.next();}
        it.next();
      }
      throw new IndexOutOfBoundsException("Entry does not exist: " + i);
    } else if (object instanceof Collection<?>) {
      final Iterator<?> iterator = ((Collection<?>) object).iterator();
      return get(iterator, i);
    } else if (object instanceof Enumeration<?>) {
        final Enumeration<?> it = (Enumeration<?>) object;
        while (it.hasMoreElements()) {
          i--;
          if (i == -1) {return it.nextElement();} 
          else {it.nextElement();}
        }
        throw new IndexOutOfBoundsException("Entry does not exist: " + i);
    } else if (object == null) {
      throw new IllegalArgumentException("Unsupported object type: null");
    } else {
      try {return Array.get(object, i);
      } catch (final IllegalArgumentException ex) {
        throw new IllegalArgumentException("Unsupported object type: " + object.getClass().getName());
      }
    }
}
// sosie  
public static Object get(final Object object, final int index) {
  int i = index;
  if (i < 0) {
    throw new IndexOutOfBoundsException("Index cannot be negative: " + i);
  }
  if (object instanceof Map<?,?>) {
    final Map<?, ?> map = (Map<?, ?>) object;
    final Iterator<?> iterator = map.entrySet().iterator();
    return get(iterator, i);
  } else if (object instanceof Object[]) {
    %*{\textbf{\color{orange}the return statement is replaced by the following}}*)
    try {
        return Array.get(object, i);
    } catch (final IllegalArgumentException ex) {
        throw new IllegalArgumentException("Unsupported object type: " + object.getClass().getName());
    }
    } else if (object instanceof Iterator<?>) {
      final Iterator<?> it = (Iterator<?>) object;
      while (it.hasNext()) {
        i--;
        if (i == -1) {return it.next();}
        it.next();
      }
      throw new IndexOutOfBoundsException("Entry does not exist: " + i);
    } else if (object instanceof Collection<?>) {
      final Iterator<?> iterator = ((Collection<?>) object).iterator();
      return get(iterator, i);
    } else if (object instanceof Enumeration<?>) {
        final Enumeration<?> it = (Enumeration<?>) object;
        while (it.hasMoreElements()) {
          i--;
          if (i == -1) {return it.nextElement();} 
          else {it.nextElement();}
        }
        throw new IndexOutOfBoundsException("Entry does not exist: " + i);
    } else if (object == null) {
      throw new IllegalArgumentException("Unsupported object type: null");
    } else {
      try {return Array.get(object, i);
      } catch (final IllegalArgumentException ex) {
        throw new IllegalArgumentException("Unsupported object type: " + object.getClass().getName());
      }
    }
}
\end{lstlisting}
\tabcolsep=0.11cm
\begin{tabular}{>{\small}c>{\small}c>{\small}c>{\small}c>{\small}c>{\small}c>{\small}c>{\small}c}
\hline
\rowcolor{lightgray} \#tc & \#assert & transfo & node & min & max & median & mean   \\
\rowcolor{lightgray}  & & type & type & depth  & depth & depth & depth  \\ 
\hline
 &  & rep &  &  &  &  & \\
\hline
\end{tabular}
\end{figure}
\section{To-be-discussed sosies}
\input{sosie002-easymock-createtypevariablemap.tex}
\input{sosie007-easymock-matches.tex}
I think this one is a good one because it just makes the behavior more conservative (no multi threading), but can this prevent some computation?

%org.easymock.internalMocksBehavior:188
\begin{minipage}{\columnwidth}
\begin{lstlisting}[caption={Two variants of \texttt{isThreadSafe} in EasyMock},language=java,numbers=left]
//original
public boolean isThreadSafe() {
  return this.isThreadSafe;
}
//sosie
public boolean isThreadSafe() {
  return false;
}
\end{lstlisting}
\tabcolsep=0.11cm
\begin{tabular}{>{\small}c>{\small}c>{\small}c>{\small}c>{\small}c>{\small}c>{\small}c>{\small}c}
\hline
\rowcolor{lightgray} \#tc & \#assert & transfo & node & min & max & median & mean   \\
\rowcolor{lightgray}  & & type & type & depth  & depth & depth & depth  \\ 
\hline
&  & rep &  &  &  &  & \\
\hline
\end{tabular}
\end{minipage}
%position: org.mozilla.javascript.ast.SwitchStatement:125
\begin{minipage}{\columnwidth}
\begin{lstlisting}[caption={\texttt{setLp} in Rhino and a sosie},language=java,numbers=left]
//original
public void setLp(int lp) {
  this.lp = lp;
}
//sosie
public void setLp(int lp) {
  this.lp = -1;
}
\end{lstlisting}
\tabcolsep=0.11cm
\begin{tabular}{>{\small}c>{\small}c>{\small}c>{\small}c>{\small}c>{\small}c>{\small}c>{\small}c}
\hline
\rowcolor{lightgray} \#tc & \#assert & transfo & node & min & max & median & mean   \\
\rowcolor{lightgray}  & & type & type & depth  & depth & depth & depth  \\ 
\hline
&  & rep &  &  &  &  & \\
\hline
\end{tabular}
\end{minipage}

\section{Bad sosies}
Listing \ref{lst:whileClosure} results from a lack of specification of the original method in the test suite. 
The original method natural specification (in comments) specifies that is must return a \texttt{Closure} that will call the closure repeatedly until the predicate returns false, but the sosie returns a closure that performs no action when the predicate is false (according to the specification of the transplant). 
The sosie clearly does not conform to the specification, but it is not captured by the test suite.

%org.apache.commons.collections4.ClosureUtils:130
\begin{figure}[ht]
\begin{lstlisting}[caption={\texttt{whileClosure} in commons.collection},label={lst:whileClosure},language=java,numbers=left]
//original program
<E> Closure<E> whileClosure(final Predicate<? super E> predicate, final Closure<? super E> closure) {
  return WhileClosure.<E>whileClosure(predicate, closure, false);
}
// sosie  
<E> Closure<E> whileClosure(final Predicate<? super E> predicate, final Closure<? super E> closure) {
  %*{\textbf{\color{orange}the return statement is replaced by the following}}*)
  return IfClosure.<E>ifClosure(predicate, trueClosure, NOPClosure.<E>nopClosure());
}
\end{lstlisting}
\tabcolsep=0.11cm
\begin{tabular}{>{\small}c>{\small}c>{\small}c>{\small}c>{\small}c>{\small}c>{\small}c>{\small}c}
\hline
\rowcolor{lightgray} \#tc & \#assert & transfo & node & min & max & median & mean   \\
\rowcolor{lightgray}  & & type & type & depth  & depth & depth & depth  \\ 
\hline
 &  & rep &  &  &  &  & \\
\hline
\end{tabular}
\end{figure}


% to add the bibliography, uncomment the following lines
% \bibliography{references}
% \bibliographystyle{plain}
\end{document}
