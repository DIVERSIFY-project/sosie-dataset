
\documentclass[table]{sig-alternate}

\usepackage[T1]{fontenc} %% to get Type 1 fonts
\usepackage[utf8]{inputenc} %% to enable non ASCII characters
%\usepackage[a4paper]{geometry} %% to change the margins

\usepackage[]{mdframed}
\usepackage[]{tabularx}
\usepackage[]{graphicx}
\usepackage[]{xcolor}
\usepackage[]{pdfcomment}
\usepackage{color}
\usepackage{listings} 
\usepackage[linesnumbered]{algorithm2e}
%\usepackage{framed}
\usepackage{paralist}
\usepackage{array}
\usepackage{numprint}
%\usepackage[table]{xcolor}

%\usepackage[font=footnotesize,labelfont=bf,justification=centering]{caption}
\usepackage[justification=centering]{caption}

\newtheorem{definition}{Definition}
\lstset{ 
  basicstyle=\ttfamily\scriptsize,        
  captionpos=t,                    
  commentstyle=\color{cyan},    
  escapeinside={\%*}{*)},          
  keywordstyle=\color{blue},       
  stringstyle=\color{red},       
  numberstyle=\tiny\color{gray}, 
  stepnumber=2,                    
  title=\lstname,                   
  language=Java,
  breakatwhitespace=false,         
  breaklines=true,                 
  extendedchars=true,              
  showspaces=false,                
  showstringspaces=false,          
  showtabs=false,                  
  tabsize=2                       
} 



%\newcommand{\TODO}[1]{\textcolor{red}{#1}\pdfcomment[color=yellow,open=false]{#1}}
\newcommand{\nbprograms}{\numprint{6}\xspace}
\newcommand{\nbsosies}{XXX\xspace}
\newcommand{\TODO}[1]{\textcolor{red}{#1}}
\newcommand{\todo}{\TODO}


% ISSTA title: Tailored Source Code Transformations to Synthesize Computationally Diverse Program Variants
\title{A dataset of program sosies}
\author{}

%Hypothesis: at every transplantation point where we can delete a statement, we can also add and replace.
%Hypothesis: 
%- there are more sosies in transplantation points that are covered by a small number of test cases.
%- there are more sosies in transplantation points that are in private methods.
%Hypothesis: the most interesting sosies are at transplantation points that are covered by a large number of test cases
%In cases of delete and replace => let us check if the AST that we delete is present elsewhere in the program.
%Are all trials in “hash*” methods succesful sosies?

\begin{document}
\maketitle
\section{Introduction}

In this document we present a dataset of sosies that have been generated for large open source Java programs. 
The objective is to illustrate the diversity of situations in which we can generate good or bad sosies, as defined in our previous work \footnote{\url{https://hal.archives-ouvertes.fr/file/index/docid/938855/filename/sosies.pdf}}. 

This dataset will grow in the future and will serve as the basis to establish a taxonomy of different code regions where we can synthesize valuable program sosies. 
\section{Good sosies}
Listing \ref{lst:hashcode} is a good example of a sosie in a function which specification is flexible, i.e., it exhibits specification diversity. 
\todo{elaborate on the expected behavior of hashcode}

% position: org.mozilla.classfile.FieldOrMethodRef:4776
\begin{minipage}{\columnwidth}
\begin{lstlisting}[caption={\texttt{hashCode} in Rhino and a sosie},label={lst:hashcode},language=java,numbers=left]
//original
public int hashCode(){}
  if (hashCode == -1) {
  int h1 = className.hashCode();
  int h2 = name.hashCode();
  int h3 = type.hashCode();
  hashCode = h1 ^ h2 ^ h3;
  }
  return hashCode;
}
//sosie
public int hashCode(){}
  if (hashCode == -1) {
  int h1 = className.length();
  int h2 = name.hashCode();
  int h3 = type.hashCode();
  hashCode = h1 ^ h2 ^ h3;
  }
  return hashCode;
}
\end{lstlisting}
\tabcolsep=0.11cm
\begin{tabular}{>{\small}c>{\small}c>{\small}c>{\small}c>{\small}c>{\small}c>{\small}c>{\small}c}
\hline
\rowcolor{lightgray} \#tc & \#assert & transfo & node & min & max & median & mean   \\
\rowcolor{lightgray}  & & type & type & depth  & depth & depth & depth  \\ 
\hline
&  & rep &  &  &  &  & \\
\hline
\end{tabular}
\end{minipage} %plastic computation
\begin{minipage}{\columnwidth}
\begin{lstlisting}[caption={\texttt{toJson} in GSON and a sosie},language=java,numbers=left]

// Original program
void toJson(Object src, Type typeOfSrc, JsonWriter writer){
    writer.setSerializeNulls(oldSerializeNulls); } }

//sosie
void toJson(Object src, Type typeOfSrc, JsonWriter writer){
    writer.setIndent("  ") 
} }
\end{lstlisting}
\tabcolsep=0.11cm
\begin{tabular}{>{\small}c>{\small}c>{\small}c>{\small}c>{\small}c>{\small}c>{\small}c>{\small}c}
\hline
\rowcolor{lightgray} \#tc & \#assert & transfo & node & min & max & median & mean   \\
\rowcolor{lightgray}  & & type & type & depth  & depth & depth & depth  \\ 
\hline
 &  & rep &  &  &  &  & \\
\hline
\end{tabular}
\end{minipage}
 %plastic computation
In listing \ref{lst:writeStringToFile}, the original program calls \texttt{openOutputStream}, which checks different things about the file name, while the sosie  directly calls the constructor of \texttt{FileOutputStream}. 
In all nominal cases, these two programs behave exactly in the same way, and the sosie executes less code. 
In exceptional cases, i.e. when \texttt{writeStringToFile()} is called with an invalid file name, the original program handles it, while the variant throws a \texttt{FileNotFoundException}. 
The original program and the sosie exhibit failure diversity on exceptional cases.

Considering the parts of the program that handle checks and exceptional cases as plasticity in the specification and the implementation is conceptually very close to the ideas of exploration of the correctness envelop as expressed by Rinard and colleagues \footnote{http://people.csail.mit.edu/rinard/paper/oopsla05.pdf}.


%org.apache.commons.io.FileUtils
\begin{minipage}{\columnwidth}
\begin{lstlisting}[caption={\texttt{writeStringToFile} in commons.io},label={lst:writeStringToFile},language=java,numbers=left]
//original program
void writeStringToFile(File file, String data, Charset encoding, boolean append) throws IOException {
  OutputStream out = null;
  out = openOutputStream(file, append); 
  IOUtils.write(data, out, encoding);
  out.close(); }

// sosie  
void writeStringToFile(File file, String data, Charset encoding, boolean append) throws IOException {
  OutputStream out = null;
  out = new FileOutputStream(file, append); //this statement replaces the original
  IOUtils.write(data, out, encoding);
  out.close(); }
\end{lstlisting}
\tabcolsep=0.11cm
\begin{tabular}{>{\small}c>{\small}c>{\small}c>{\small}c>{\small}c>{\small}c>{\small}c>{\small}c}
\hline
\rowcolor{lightgray} \#tc & \#assert & transfo & node & min & max & median & mean   \\
\rowcolor{lightgray}  & & type & type & depth  & depth & depth & depth  \\ 
\hline
 &  & rep &  &  &  &  & \\
\hline
\end{tabular}
\end{minipage}
 %remove check
Listing \ref{lst:setAttributes} is an example of sosie that removes checks on inputs.

%position: org.mozilla.javascript.IdScriptableObject:431
\begin{minipage}{\columnwidth}
\begin{lstlisting}[caption={\texttt{setAttributes} in Rhino and a sosie},label=lst:setAttributes,language=java,numbers=left]
//original
public void setAttributes(String name, int attributes){
  ScriptableObject.checkValidAttributes(attributes);
  %* {\color{gray} \emph{...} } *)
}
//sosie
public void setAttributes(String name, int attributes){
  while (attributes < attributes) {
    ++attributes;
    attributes <<= 1;
  }
  %* {\color{gray} \emph{...} } *)
}\end{lstlisting}
\tabcolsep=0.11cm
\begin{tabular}{>{\small}c>{\small}c>{\small}c>{\small}c>{\small}c>{\small}c>{\small}c>{\small}c}
\hline
\rowcolor{lightgray} \#tc & \#assert & transfo & node & min & max & median & mean   \\
\rowcolor{lightgray}  & & type & type & depth  & depth & depth & depth  \\ 
\hline
&  & rep &  &  &  &  & \\
\hline
\end{tabular}
\end{minipage}


 %remove check
Listing \ref{lst:stopmethod} is an example of sosie that hits a rigidity: assigning 0 or 4 to \texttt{itsMaxStack} has no incidence on the behavior of Rhino.


%position: org.mozilla.classfile.ClassFileWriter:438
\begin{minipage}{\columnwidth}
\begin{lstlisting}[caption={\texttt{stopMethod} in Rhino and a sosie},label={lst:stopmethod},language=java,numbers=left]
//original
public void stopMethod(short maxLocals) {
  %* {\color{gray} \emph{...} } *)
  itsMaxStack = 0;
  itsStackTop = 0;
  itsLabelTableTop = 0;
  itsFixupTableTop = 0;
  itsVarDescriptors = null;
  itsSuperBlockStarts = null;
  itsSuperBlockStartsTop = 0;
  itsJumpFroms = null;
}
//sosie
public void stopMethod(short maxLocals) {
  %* {\color{gray} \emph{...} } *)
  itsMaxStack = 4;
  itsStackTop = 0;
  itsLabelTableTop = 0;
  itsFixupTableTop = 0;
  itsVarDescriptors = null;
  itsSuperBlockStarts = null;
  itsSuperBlockStartsTop = 0;
  itsJumpFroms = null;
}
%* {\color{gray} \emph{...} } *)
\end{lstlisting}
\tabcolsep=0.11cm
\begin{tabular}{>{\small}c>{\small}c>{\small}c>{\small}c>{\small}c>{\small}c>{\small}c>{\small}c}
\hline
\rowcolor{lightgray} \#tc & \#assert & transfo & node & min & max & median & mean   \\
\rowcolor{lightgray}  & & type & type & depth  & depth & depth & depth  \\ 
\hline
&  & rep &  &  &  &  & \\
\hline
\end{tabular}
\end{minipage} %hits a rigidity
Listing \ref{lst:convertarg} is a beautiful sosie, which reduces the output space of the program, increasing safety.

%position: rg.mozilla.javascript.FunctionObject:205
\begin{minipage}{\columnwidth}
\begin{lstlisting}[caption={\texttt{convertArg} in Rhino and a sosie},label={lst:convertarg},language=java,numbers=left]
//original
public static Object convertArg(Context cx, Scriptable scope, Object arg, int typeTag){
  switch (typeTag) {
    %* {\color{gray} \emph{...} } *)
    case JAVA_OBJECT_TYPE:
      return arg;
    default:
        throw new IllegalArgumentException();
  }
}
//sosie
public static Object convertArg(Context cx, Scriptable scope, Object arg, int typeTag){
  switch (typeTag) {
    %* {\color{gray} \emph{...} } *)
    case JAVA_OBJECT_TYPE:
      if (arg != null){return arg;}
    default:
      throw new IllegalArgumentException();
  }
}

\end{lstlisting}
\tabcolsep=0.11cm
\begin{tabular}{>{\small}c>{\small}c>{\small}c>{\small}c>{\small}c>{\small}c>{\small}c>{\small}c}
\hline
\rowcolor{lightgray} \#tc & \#assert & transfo & node & min & max & median & mean   \\
\rowcolor{lightgray}  & & type & type & depth  & depth & depth & depth  \\ 
\hline
&  & rep &  &  &  &  & \\
\hline
\end{tabular}
\end{minipage} %reduces the output space
\begin{minipage}{\columnwidth}
\begin{lstlisting}[caption={\texttt{decode} in commons.codec and a sosie},language=java,numbers=left]
// Original program
void decode(final byte[] in, int inPos, final int inAvail, final Context context) {
  switch (context.modulus) {
    case 0 : // impossible, as excluded above
    case 1 : // 6 bits - ignore entirely
        // not currently tested; perhaps it is impossible?
             break;
}

// sosie
void decode(final byte[] in, int inPos, final int inAvail, final Context context) {
  switch (context.modulus) {
    case 0 : // impossible, as excluded above
    case 1 : 
}
\end{lstlisting}
\tabcolsep=0.11cm
\begin{tabular}{>{\small}c>{\small}c>{\small}c>{\small}c>{\small}c>{\small}c>{\small}c>{\small}c}
\hline
\rowcolor{lightgray} \#tc & \#assert & transfo & node & min & max & median & mean   \\
\rowcolor{lightgray}  & & type & type & depth  & depth & depth & depth  \\ 
\hline
 &  & del &  &  &  &  & \\
\hline
\end{tabular}
\end{minipage}
%CommonMath-Array2DRowRealMatrix.java:553
\begin{minipage}{\columnwidth}
\begin{lstlisting}[caption={Two variants of \texttt{getEntry} in commons.maths},language=java,numbers=left]
// Original program
public double getEntry(final int row, final int column) throws OutOfRangeException {
  MatrixUtils.checkMatrixIndex(this, row, column);
  int n = 1; //Transplant
  return data[row][column];
}

// sosie
public double getEntry(final int row, final int column) throws OutOfRangeException {
  int n = 1; //Transplant
  return data[row][column];
}
\end{lstlisting}
\tabcolsep=0.11cm
\begin{tabular}{>{\small}c>{\small}c>{\small}c>{\small}c>{\small}c>{\small}c>{\small}c>{\small}c}
\hline
\rowcolor{lightgray} \#tc & \#assert & transfo & node & min & max & median & mean   \\
\rowcolor{lightgray}  & & type & type & depth  & depth & depth & depth  \\ 
\hline
 &  & del &  &  &  &  & \\
\hline
\end{tabular}
\end{minipage}
%CommonMath-AbstractWell.java:171
\begin{minipage}{\columnwidth}
\begin{lstlisting}[caption={Two variants of \texttt{setSeed} in commons.maths},language=java,numbers=left]
// Original program
public void setSeed(final int[] seed) {
  if (seed == null) {
    setSeed(System.currentTimeMillis() + System.identityHashCode(this));
    return;
  }
  System.arraycopy(seed, 0, v, 0, FastMath.min(seed.length, v.length));
  if (seed.length < v.length) {
   for (int i = seed.length; i < v.length; ++i) {
    final long l = v[i - seed.length];
    v[i] = (int) ((1812433253l * (l ^ (l >> 30)) + i) & 0xffffffffL);
    }
  }
  index = 0;
  clear();
}
//sosie
public void setSeed(final int[] seed) {
  if (seed == null) {
    setSeed(System.currentTimeMillis() + System.identityHashCode(this));
    return;
  }
  System.arraycopy(seed, 0, v, 0, FastMath.min(seed.length, v.length));
  if (seed.length < v.length) {
   for (int i = seed.length; i < v.length; ++i) {
    final long l = v[i - seed.length];
    v[i] = (int) ((1812433253l * (l ^ (l >> 30)) + i) & 0xffffffffL);
    }
  }
  index = 0;

}
\end{lstlisting}
\tabcolsep=0.11cm
\begin{tabular}{>{\small}c>{\small}c>{\small}c>{\small}c>{\small}c>{\small}c>{\small}c>{\small}c}
\hline
\rowcolor{lightgray} \#tc & \#assert & transfo & node & min & max & median & mean   \\
\rowcolor{lightgray}  & & type & type & depth  & depth & depth & depth  \\ 
\hline
&  & del &  &  &  &  & \\
\hline
\end{tabular}
\end{minipage}
%EasyMock 3.2-ReplayState.java:38
\begin{minipage}{\columnwidth}
\begin{lstlisting}[caption={Two variants of \texttt{invoke} in EasyMock},language=java,numbers=left]
//original
public Object invoke(final Invocation invocation) throws Throwable {
  behavior.checkThreadSafety();
  if (behavior.isThreadSafe()) {
    lock.lock();
    try {
      return invokeInner(invocation);
    } finally {
      lock.unlock();
    }
  }
  return invokeInner(invocation);
}
//sosie
public Object invoke(final Invocation invocation) throws Throwable {
  behavior.checkThreadSafety();
  if (behavior.isThreadSafe()) {
    new locks.ReentrantLock();
  }
  return invokeInner(invocation);
}
\end{lstlisting}
\tabcolsep=0.11cm
\begin{tabular}{>{\small}c>{\small}c>{\small}c>{\small}c>{\small}c>{\small}c>{\small}c>{\small}c}
\hline
\rowcolor{lightgray} \#tc & \#assert & transfo & node & min & max & median & mean   \\
\rowcolor{lightgray}  & & type & type & depth  & depth & depth & depth  \\ 
\hline
&  & rep &  &  &  &  & \\
\hline
\end{tabular}
\end{minipage}
In listing \ref{lst:getCharIgnoreLineEnd}, the variable \texttt{c} is never equal to  `r' (maybe different on windows) 

%position: org.mozilla.javascript.TokenStream:1359
\begin{minipage}{\columnwidth}
\begin{lstlisting}[caption={\texttt{getCharIgnoreLineEnd} in Rhino and a sosie},label={lst:getCharIgnoreLineEnd},language=java,numbers=left]
//original
private int getCharIgnoreLineEnd() throws IOException{
  %* {\color{gray} \emph{...} } *)
  if (c <= 127) {
    if (c == '\n' || c == '\r') {
      lineEndChar = c;
      c = '\n';
    }
  } else {
  %* {\color{gray} \emph{...} } *)
}
private int getCharIgnoreLineEnd() throws IOException{
  %* {\color{gray} \emph{...} } *)
  if (c <= 127) {
    if (c == '\n' || c == '\r') {
      lineEndChar = c;
      c = (c) > c ? c : c;
      }
  } else {
  %* {\color{gray} \emph{...} } *)
}
\end{lstlisting}
\tabcolsep=0.11cm
\begin{tabular}{>{\small}c>{\small}c>{\small}c>{\small}c>{\small}c>{\small}c>{\small}c>{\small}c}
\hline
\rowcolor{lightgray} \#tc & \#assert & transfo & node & min & max & median & mean   \\
\rowcolor{lightgray}  & & type & type & depth  & depth & depth & depth  \\ 
\hline
&  & rep &  &  &  &  & \\
\hline
\end{tabular}
\end{minipage}



\section{Useless sosies}
\label{sec:useless}
%EasyMock 3.2-Capture:131
\begin{minipage}{\columnwidth}
\begin{lstlisting}[caption={\texttt{toString} in EasyMock and a sosie},language=java,numbers=left]
//original
public String toString() {
  if (values.isEmpty()) {
    return "Nothing captured yet";
  }
  if (values.size() == 1) {
    return String.valueOf(values.get(0));
  }
  return values.toString();
}
//sosie
public String toString() {
  if (values.isEmpty()) {
    return "Nothing captured yet";
  }
  if (values.size() == 1) {
    if ((values.size()) == 1) {
      return String.valueOf(values.get(0));
    }
  }
  return values.toString();
}
\end{lstlisting}
\tabcolsep=0.11cm
\begin{tabular}{>{\small}c>{\small}c>{\small}c>{\small}c>{\small}c>{\small}c>{\small}c>{\small}c}
\hline
\rowcolor{lightgray} \#tc & \#assert & transfo & node & min & max & median & mean   \\
\rowcolor{lightgray}  & & type & type & depth  & depth & depth & depth  \\ 
\hline
&  & rep &  &  &  &  & \\
\hline
\end{tabular}
\end{minipage} %adds redundant code
Listing \ref{lst:ensureIndex} is a valid sosie, not beautiful: the first assignment of \texttt{index} is useless

%Position: org.mozilla.javascript.UintMap:290
\begin{minipage}{\columnwidth}
\begin{lstlisting}[caption={\texttt{ensureIndex} in Rhino and a sosie},label=lst:ensureIndex,language=java,numbers=left]
//original
private int ensureIndex(int key, boolean intType) {
  int index = -1;
  //end transformation
  int firstDeleted = -1;
  int[] keys = this.keys;
  if (keys != null) {
    int fraction = key * A;
    index = fraction >>> (32 - power);
    %* {\color{gray} \emph{...} } *)
    }
    // Inserting of new key
    if (check && keys != null && keys[index] != EMPTY)
        Kit.codeBug();
    if (firstDeleted >= 0) {
        index = firstDeleted;
    }
    else {
      // Need to consume empty entry: check occupation level
      if (keys == null || occupiedCount * 4 >= (1 << power) * 3) {
        // Too litle unused entries: rehash
        rehashTable(intType);
        return insertNewKey(key);
      }
    ++occupiedCount;
  }
  keys[index] = key;
  ++keyCount;
  return index;
}
//sosie
private int ensureIndex(int key, boolean intType) {
  int index = 65536;
  //end transformation
  int firstDeleted = -1;
  int[] keys = this.keys;
  if (keys != null) {
    int fraction = key * A;
    index = fraction >>> (32 - power);
    %* {\color{gray} \emph{...} } *)
    }
    // Inserting of new key
    if (check && keys != null && keys[index] != EMPTY)
        Kit.codeBug();
    if (firstDeleted >= 0) {
        index = firstDeleted;
    }
    else {
      // Need to consume empty entry: check occupation level
      if (keys == null || occupiedCount * 4 >= (1 << power) * 3) {
        // Too litle unused entries: rehash
        rehashTable(intType);
        return insertNewKey(key);
      }
    ++occupiedCount;
  }
  keys[index] = key;
  ++keyCount;
  return index;
}
\end{lstlisting}
\tabcolsep=0.11cm
\begin{tabular}{>{\small}c>{\small}c>{\small}c>{\small}c>{\small}c>{\small}c>{\small}c>{\small}c}
\hline
\rowcolor{lightgray} \#tc & \#assert & transfo & node & min & max & median & mean   \\
\rowcolor{lightgray}  & & type & type & depth  & depth & depth & depth  \\ 
\hline
&  & rep &  &  &  &  & \\
\hline
\end{tabular}
\end{minipage}
\section{To-be-discussed sosies}
%EasyMock 3.2-BridgeMethodResolver:254
\begin{minipage}{\columnwidth}
\begin{lstlisting}[caption={\texttt{createTypeVariableMap and a sosie} in EasyMock},language=java,numbers=left]
// Original program
private static Map<TypeVariable<?>, Type> createTypeVariableMap(final Class<?> cls) {
  final Map<TypeVariable<?>, Type> typeVariableMap = new HashMap<TypeVariable<?>, Type>();
  extractTypeVariablesFromGenericInterfaces(cls.getGenericInterfaces(), typeVariableMap);
  Type genericType = cls.getGenericSuperclass();
  Class<?> type = cls.getSuperclass();
  while (!Object.class.equals(type)) {
    if (genericType instanceof ParameterizedType) {
      final ParameterizedType pt = (ParameterizedType) genericType;
      populateTypeMapFromParameterizedType(pt, typeVariableMap);
      }
      extractTypeVariablesFromGenericInterfaces(type.getGenericInterfaces(), typeVariableMap);
      genericType = type.getGenericSuperclass();
      type = type.getSuperclass();
    }
    type = cls;
    while (type.isMemberClass()) {
      genericType = type.getGenericSuperclass();
      if (genericType instanceof ParameterizedType) {
        final ParameterizedType pt = (ParameterizedType) genericType;
        populateTypeMapFromParameterizedType(pt, typeVariableMap);
      }
      type = type.getEnclosingClass();
    }                 
  return typeVariableMap;
}
//sosie
private static Map<TypeVariable<?>, Type> createTypeVariableMap(final Class<?> cls) {
  final Map<TypeVariable<?>, Type> typeVariableMap = new HashMap<TypeVariable<?>, Type>();
  extractTypeVariablesFromGenericInterfaces(cls.getGenericInterfaces(), typeVariableMap);
  Type genericType = cls.getGenericSuperclass();
  Class<?> type = cls.getSuperclass();
  while (!Object.class.equals(type)) {
    if (genericType instanceof ParameterizedType) {
      final ParameterizedType pt = (ParameterizedType) genericType;
      populateTypeMapFromParameterizedType(pt, typeVariableMap);
      }
      extractTypeVariablesFromGenericInterfaces(type.getGenericInterfaces(), typeVariableMap);
      genericType = type.getGenericSuperclass();
      type = type.getSuperclass();
    }
    type = cls;
  return typeVariableMap;
}
\end{lstlisting}
\tabcolsep=0.11cm
\begin{tabular}{>{\small}c>{\small}c>{\small}c>{\small}c>{\small}c>{\small}c>{\small}c>{\small}c}
\hline
\rowcolor{lightgray} \#tc & \#assert & transfo & node & min & max & median & mean   \\
\rowcolor{lightgray}  & & type & type & depth  & depth & depth & depth  \\ 
\hline
&  & del &  &  &  &  & \\
\hline
\end{tabular}
\end{minipage}
%easymock.internal.matchers.equals:39
\begin{minipage}{\columnwidth}
\begin{lstlisting}[caption={Two variants of \texttt{matches} in EasyMock},language=java,numbers=left]
//original
public boolean matches(final Object actual) {
  if (this.expected == null) {
    return actual == null;
  }
  return expected.equals(actual);
}
//sosie
public boolean matches(final Object actual) {
  if (this.expected == null) {
    return true;
  }
  return expected.equals(actual);
}
\end{lstlisting}
\tabcolsep=0.11cm
\begin{tabular}{>{\small}c>{\small}c>{\small}c>{\small}c>{\small}c>{\small}c>{\small}c>{\small}c}
\hline
\rowcolor{lightgray} \#tc & \#assert & transfo & node & min & max & median & mean   \\
\rowcolor{lightgray}  & & type & type & depth  & depth & depth & depth  \\ 
\hline
&  & rep &  &  &  &  & \\
\hline
\end{tabular}
\end{minipage}
I think this one is a good one because it just makes the behavior more conservative (no multi threading), but can this prevent some computation?

%org.easymock.internalMocksBehavior:188
\begin{minipage}{\columnwidth}
\begin{lstlisting}[caption={Two variants of \texttt{isThreadSafe} in EasyMock},language=java,numbers=left]
//original
public boolean isThreadSafe() {
  return this.isThreadSafe;
}
//sosie
public boolean isThreadSafe() {
  return false;
}
\end{lstlisting}
\tabcolsep=0.11cm
\begin{tabular}{>{\small}c>{\small}c>{\small}c>{\small}c>{\small}c>{\small}c>{\small}c>{\small}c}
\hline
\rowcolor{lightgray} \#tc & \#assert & transfo & node & min & max & median & mean   \\
\rowcolor{lightgray}  & & type & type & depth  & depth & depth & depth  \\ 
\hline
&  & rep &  &  &  &  & \\
\hline
\end{tabular}
\end{minipage}
%position: org.mozilla.javascript.ast.SwitchStatement:125
\begin{minipage}{\columnwidth}
\begin{lstlisting}[caption={\texttt{setLp} in Rhino and a sosie},language=java,numbers=left]
//original
public void setLp(int lp) {
  this.lp = lp;
}
//sosie
public void setLp(int lp) {
  this.lp = -1;
}
\end{lstlisting}
\tabcolsep=0.11cm
\begin{tabular}{>{\small}c>{\small}c>{\small}c>{\small}c>{\small}c>{\small}c>{\small}c>{\small}c}
\hline
\rowcolor{lightgray} \#tc & \#assert & transfo & node & min & max & median & mean   \\
\rowcolor{lightgray}  & & type & type & depth  & depth & depth & depth  \\ 
\hline
&  & rep &  &  &  &  & \\
\hline
\end{tabular}
\end{minipage}
\section{Bad sosies}


% to add the bibliography, uncomment the following lines
% \bibliography{references}
% \bibliographystyle{plain}
\end{document}
